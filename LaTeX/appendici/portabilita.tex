\section{Ambiente di sviluppo}
Inizialmente si è optato per lavorare in un sistema operativo Linux x86, ma a causa di numerosi problemi nella compilazione di OpenCV il laureando si è spostato in ambiente Windows 7 x86. Il codice è stato sviluppato, testato e compilato in {\em Visual Studio 2010}.
L'elemento che ha influenzato maggiormente la scelta è la possibilità del linking del proprio codice con numerose dll (dynamic linked library) esterne anzichè dover compilarne i sorgenti da zero. Questo chiaramente rende il codice non portabile fuori dall'ambiente Windows. \\

Headers propri di Visual Studio inoltre sono stati utilizzati per il calcolo dell'fps in acquisizione. Questo dato è fondamentale per il processing, perchè ricordiamo esso riveste le serie di una veste temporale precisa. Qualora lo sviluppo del codice prosegua fuori da Windows, si dovrà ricorrere a diversi metodi di accesso ai tempi macchina. \\

L'interfacciamento con Matlab consta di un header da includere e tre librerie {\em lib} da collegare. Sebbene illustrato a prescindere dal sistema operativo \cite{MATLABC}, esso è stato testato solo in Windows e con una versione Matlab ``portable''. Durante la compilazione è possibile però escludere Matlab, rinunciando ai grafici di output e a un algoritmo di terze parti \cite{CARD} per l'ICA, al cui posto verrebbe compilato del codice analogo ma non testato completamente.