Qui è elencata una serie di parametri macro da specificare a tempo di compilazione.
In riscritture future del codice è considerabile l'idea di portarne almeno una parte a run-time mediante allocazione dinamica della memoria; questa non è però rientrata negli obiettivi di first-release che il laureando si è posto.
\mylist{MATLAB: se definito, consente l'appoggio esterno a Matlab.; CAMERA: se definito, inizia una cattura da webcamera, altrimenti apre un file AVI.; WINTIMESTAMP: se definito, utilizza funzioni per il calcolo del {\em cputime} testate solo in {\em Visual Studio}.; DONOTCLOSEENGINE (richiede MATLAB): se definito, al termine del processo non chiama la chiusura del Matlab Engine per velocizzare le esecuzioni future.; RES\_1280\_720 (richiede CAMERA): se definito, prova a imporre l'alta risoluzione alla webcam, altrimenti, o in caso di fallimento del tentativo, verrà usata quella di default del device. Fattore di onerosità nel calcolo e nella memoria.; SRC\_FACE: se definito, predispone l'acquisizione da regione facciale e il riconoscimento tramite HAAR. Deprecato, e probabilmente guasto.; FRAMEBLOCK: intero positivo per il numero di frame contigui che formino un blocco di processing. Fattore di onerosità nel calcolo e nella memoria.; OVERLAP: intero positivo minore di FRAMEBLOCK per l'acquisizione parziale di nuovi frames.; DETREND: se definito, considera la fase di detrending.; LAMBDA: (richiede DETREND): parametro regolatore dell'algoritmo di detrending.; VERBOSE: flag per il livello di verbosità del programma nella comunicazione del proprio operato.; HUNGRY: flag per la rigorosità delle deallocazioni in memoria. Più alto, più severo.; AVGPOINTS: punti per il filtro MA.; BPPOINTS: punti per il filtro BP.; INTERPFREQ: frequenza alla cui interpolare.}